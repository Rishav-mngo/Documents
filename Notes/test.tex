 \documentclass{article}
 \usepackage[utf8]{inputenc}
 \usepackage[T1]{fontenc}
 
 \title{notes}
 \author{Rishav Raj}
 \date{May 2020}
 
 \begin{document}
 
 
 
 \section*{PREFACE}
 
 \begin{flushright}In computing, turning the obvious into the useful \\ is a living definition of the word "frustration." \end{flushright}
 

 
 In the years since the first edition of C Programming: A Modern Approach was published, a host of new c-based languages have sprung up---Java and c\# foremost among them---an
 d related languages such as C++
 and perl have achieved greater prominence. Still , C remains as popular as ever, plugging away in the background,quietly powering much of the world's software. It remains t
 he lingua fraca of the computer universe, as it was in 1996. \\
    But even C must change with the times. The need for a new edition of C Programming: A Modern Approch became apparent when the C99 standard was published. More ever, the f
 first edition, with its references to DOS and 16-bit processors, was becoming dated. The second edition is fully up-to-date and has been improved in many other ways as well.
 \\
    \\
     \\
    \textbf{What's New in the Second Edition}\\
    \\
    Here's a list of new features and improvements in the second edition:
 \begin{itemize}
     \item \textbf{Complete coverage of both the C89 standard and the c99 standard. } The biggest difference between the first and second editions is coverage of the C99 standard
 . My goal was to cover every significant difference between C89 and C99, including all the language features and library functions added in C99. Each C99 change is clearly marked, either with "C99" in the heading of a section or-- in the Case of shorter discussions--with a special icon in the left margin. I did the partly to draw attention to t
 he changes and partly so the readers who aren't interested in C99 or don't have access to a C99 compiler will know what to skip. Many of the C99 additions are of interest only to specialized audience, but some of the new features will be of use to nearly all C programmers.
 \\
 \\
     \item\textbf{ Includes a quick reference to all C89 and C99 library functions.} Appendix D in the first edition described all C89 standard library functions. In this edition, the appendix covers all C89 and C99 library functions. 
     \\
     \\
     \item\textbf{Expanded coverage of GCC.} In the years since the first edition, Use of GCC (originally the GNU Compiler Collection) has spread. GCC has some significant advantages, including high quality, low(i.e.,no) cost, and portability across a variety of hardware and software platforms. In recognition of its growing importance, I've included more information about GCC in this edition, including discussions of how to use is as well as common GCC error messages and warning.
     \\
     \\
     \item\textbf{New coverage of abstract add types.} In the first edition, a significant portion of Chapter 19 was devoted to C++. This material seems less relevant today, since students may already have learned C++ , Java, or C# before reading this book IN this edition , coverage of 
     C++ has been replaced by a discussion of how to set up abstract data types in C.
     \\
     \\
     \item\textbf{Expanded coverage of international features.} Chapter 25, which is devoted to C
     s international features, is now much longer and more detailed. Information about the Unicode/UCS Character set and its encoding is a highlight of  the expanded coverage.
     \\
     \\
     \item\textbf{Updated to reflect today's CPUs and operating systems.} When I Wrote the first edition, 16-bit architectures and the DOS operating system were still relevant to focus more on 32-bit and 64-bit architectures. The rise of Linux and other versions of UNIX has dictated a stronger focus on that family of operating systems, although aspects of Windows and the Mac OS operating system that affect C programmers are mentioned as well.
     \\
     \\
     \item\textbf{More exercise and programming projects.} The first edition of this book contained 311 exercises. This edition has nearly 500 (498, to be exact), divided into two groups: exercises and programming projects.
     \\
     \\
      \item\textbf{Solutions to selected exercise and programming projects.} The most frequent request I received from readers of the first edition was to provide answers to the exercise. In response to this request, I've put the answers to roughly one-third of the exercise an programming projects on the web at knkning.com/books/c2. This feature is particularly useful for readers who aren't enrolled in a collage course and need a way to check their work. Exercise and projects for which answers are provided are marked with a W icon (the "W" stands for "answer available on the Web").
      \\
      \\
       \item\textbf{Password-protected instructor website.} For this edition, I've built a new instructor resource site (accessible through Knking.com/books/c2) containing solutions to the remaining exercise and projects, plus PowerPoint presentations for most chapter. Faculty may contact me at cbook@Kniking.com for a password. Please us your campus email address and include a link to your department's website so that I can verify your identity.
       \\
       \\
       I've also taken the opportunity to improve wording and explanations throughout the book. The changes are extensive an painstaking: every sentence has been checked ans --if necessary--rewritten.\\
        Although much has changed in this edition, I've tried to retain the original chapter and section numbering as much as possible. Only one chapter (the last one) is entirely new, but many chapters have additional sections. In a few cases, existing sections have been renumbered. One appendix (C syntax) has been dropped, but a new appendix that compares C99
        with C89 has been added.
        \\
        \\
 \end{itemize}
 \end{document}
                                                                                                                                                                             
                                                                                                                                                                             
                                                                                                                                                                             
                                                                                                                                                                             
                                                                                                                                                                             

